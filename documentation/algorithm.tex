\documentclass[12px]{article}

% Algorithmic packages
\usepackage{amsmath}
\usepackage{amsthm}
\usepackage{algorithm}
\usepackage[noend]{algpseudocode}
\usepackage[a4paper]{geometry}

\title{Distributed REM Algorithm}
\author{Fredrik Manne, Rémi Dupré}

\newtheorem{lemma}{Lemma}


\begin{document}
  \maketitle

  \section{Algorithm}
  Assumptions:
  \begin{itemize}
    \item $\forall x, p(x) \leq x$.
    \item Each process can determine the owner of any node.
  \end{itemize}

  \begin{algorithm}
    \caption{Handle one request $(r_x, r_y)$ on $owner(r_x)$}
    \begin{algorithmic}
      \State $r_x \gets local\_root(r_x)$
      \State
      \If {$p(r_x) < r_y$}
        \State request $owner(r_y)$ with parameter $(r_y, p(r_x))$
      \ElsIf {$p(r_x) > r_y$}
        \If {$p(r_x) = r_x$}
          \State $p(r_x) \gets r_y$
        \Else
          \State $z \gets p(r_x)$
          \State $p(r_x) \gets r_y$
          \State request $owner(z)$ with parameter $(z, r_y)$
        \EndIf
      \EndIf
    \end{algorithmic}
    \label{algo:handle_task}
  \end{algorithm}


  \section{Correctness}

  \subsection{Notations}
    In order to make the proof clearer, some of the structures will be simplified.

    \paragraph{dset} is the forest holding the disjoint set structure representing $p$. Thus we can use two kind of notations: $p(x) = y \Leftrightarrow (x, y) \in dset \text{ and } (y, x) \in dset$. Notice that as vertices are partitioned between process, it is not necessary to split this structure into one per process, in the actual algorithm only the process owning $x$ knows the value of $p(x)$.

    \paragraph{tasks} is a graph containing all the pairs $(r_x, r_y)$ the algorithm needs to handle.

    \paragraph{union\_graph} is the graph of all edges that the algorithm still kept memory of. $union\_graph = dset \cup tasks$.

    \paragraph{equivalence} a relation $\sim$ is defined as $G_1$ $\sim$ $G_2$ $\Leftrightarrow$ $G_1$ and $G_2$ have the same components. \\

    A step of the algorithm consists in poping a task $(r_x, r_y)$ from the graph $tasks$ and then by applying algorithm \ref{algo:handle_task}, the states in $dset$ and $tasks$ is modified.
    Thus we can denotate the state of the algorithm after $t$ iterations by $dset_t$ and $tasks_t$.

  \subsection{Proof}
    \begin{lemma}
      At any step $t$ of algorithm \ref{algo:handle_task}, $union\_graph_t \sim union\_graph_{t+1}$.
    \end{lemma}
    \begin{proof}
      At any t, the algorithm pops the edge $(r_x, r_y) \in tasks_t$. We need to check that $r_x$ and $r_y$ are still in the same component of $union\_graph$ and that no other component is merged.
      \begin{itemize}
        \item If $p_t(r_x) = r_y$, then $(r_x, r_y) \in dset_t = dset_{t+1}$, then $union\_graph_t = union\_graph_{t+1}$.
        \item If $p_t(r_x) < r_y$, then $(r_y, p_t(r_x))$ is inserted in $tasks$. Thus on any path, the edge $(r_x, r_y) \in tasks_t$ can be replace by edges $(r_x, p_t(r_x)) \in dset_t = dset_{t+1}$ and $(p_t(r_x), r_y) \in tasks_{t+1}$.
        \item If $p_t(r_x) > r_y$:
          \begin{itemize}
            \item If $p(r_x) = r_x$, then $(r_x, r_y)$ is added to $dset_{t+1}$, then $union\_graph_t = union\_graph_{t+1}$.
            \item If $p(r_x) \neq r_x$, $(p_{t+1}(r_x) = r_y)$ and $(p_t(r_x), r_y)$ is added to $tasks_{t+1}$. All theses nodes were part of the same component in $union\_graph_t$, no components are merged. Also, $(r_x, r_y)$ is added to $dset_{t+1}$, not path using this edge is broken. Finaly, there is a new path from $r_x$ to $p_t(r_x)$: $(r_x, r_y)(r_y, p_t(r_x))$ with $(r_x, r_y) \in dset_{t+1}$ and $(r_y, p_t(r_x)) \in tasks_{t+1}$.
          \end{itemize}
      \end{itemize}
    \end{proof}

\end{document}
